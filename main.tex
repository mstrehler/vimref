\documentclass{scrartcl}

\usepackage[utf8]{inputenc}
\usepackage[T1]{fontenc}
\usepackage[ngerman]{babel}
\usepackage{multicol}
\usepackage{mdwlist} % für kompakte Listendarstellung

\begin{document}
\section*{Vim Ref}

\begin{multicols}{2}

\subsection*{Bewegungs-Befehle}

\begin{description*}
	\item[w,b] Wort vor, zurück
	\item[W,B] WORT vor, zurück
	\item[0,\$] Zum Anfang,Ende der Zeile gehen
	\item[i,a] Einfügen vor, nach aktuellem Zeichen 
	\item[I,A] Einfügen am Anfang (ex. Space), Ende der Zeile
	\item[gg,G] Zur Zeile <Nr> gehen (Anfang des Files), zum Ende des Files gehen. 	  
\end{description*} 

\subsection*{Makros}

\begin{description*}
  \item[qa,qA] Makroaufnahme starten: Reg. <<a>> überschreiben, an Reg. <<a>> anhängen.
   \item[q] Makroaufnahme stoppen
   \item[@a,@@] Makro Reg. <<a>> starten, letztes Makro starten.
   \item[:reg a] Inhalt Reg. <<a>> anzeigen.
\end{description*}

\subsection*{Vim-Kommandos}

\begin{description*}

	\item[:e <Datei>] öffnet Datei (ohne Anführungs u. Schlusszeichen)
	\item[:3d] lösche die dritte Zeile
	\item[:3,5d] lösche Zeile 3 bis 5
	\item[:.d] lösche aktuelle Zeile (.)
	\item[. \$ \%] aktuelle Zeile, letzte Zeile, alle Zeilen
   \item[:nohlsearch] Löscht Hervorhebungen (nach Suchvorgängne)
\end{description*} 

\subsection*{Zeilenweise Kommandos}

\begin{description*}

	\item[yy] Zeile ins yank-Register kopieren (Reg. \dq)
	\item[yyp] Zeile duplizieren
	\item[dd] aktuelle Zeile löschen
	\item[ddp] Zeile nach unten verschieben (delete und put)
	\item[o,O] eine Zeile unten, oben einfügen und in den Insert-Modus wechseln.
	\item[0,\$] zum Anfang,Ende der Zeile gehen (0=Null)
	\item[I,A] zum Anfang,Ende der Zeile gehen u. in den Insert-Modus wechseln.

\end{description*}

\subsection*{Suchen und Ersetzen}

\begin{description*}

  \item[:\%s/<suchen>/<ersetzen>/] Lokales Ersetzen
  \item[:\%s/<suchen>/<ersetzen>/g] globales Ersetzen
  \item[\textbackslash c,\textbackslash C] schaltet Option \texttt{ignorecase} ein,aus. 
  \item[\textbackslash v,\textbackslash V] schaltet Option \texttt{very magic} ein, wörtliche Suche


\end{description*}


\subsection*{Split Windows}

\begin{description*}

  \item[:split <Datei>] Lädt <Datei> und teilt das Fenster horizontal (eins oben, eins unten)
  \item[:vsplit <Datei>] Lädt <Datei> und teilt das Fenster vertikal
  \item[\ldots] Wechselt zwischen den Fenster

\end{description*}

\subsection*{Tabulator}

  \begin{description*}

    \item[:tabe <Datei>] Lädt <Datei> und erzeugt ein neuen Tabulator
    \item[gt, gT, 5gt] nächster, vorheriger Tabulator, zum 5. Tabulator gehen

  \end{description*}

\subsection*{Buffer}

\begin{description*}

  \item[:buffer <Datei>] Lädt <Datei> und erzeugt einen neuen Buffer.
  \item[:bn] zum nächsten Puffer gehen.

\end{description*}

\subsection*{Argumentliste}

\begin{description*}

	\item[:argdo] Ex-Befehl auf allen Elementen der Argumentliste ausführen
	\item[:args <Files>] Alle Dateien anzeigen oder mit Dateien füllen
	\item[*,**] Jokerzeichen für null od. mehrere Zeichen, ** rekursiv
	\item[**/*.tex] Globs für alle LaTeX-Dateien
	\item[:next,:prev] Argumentliste durchlaufen
\end{description*}<++>
\subsection*{Kommandos Vim-LaTeX}

\begin{description*}

  \item[:TTemplate] Zeigt Tex-Templates an (im Ordner .vim/ftplugin/latex-suite/templates)
\end{description*}

\subsection*{Optionen ein- und ausschalten}

\begin{description*}
  \item[:se[t] <Opt>] Option einschalten, ohne Parameter: alle eingeschalteten Optionen anzeigen.
  \item[:no<Opt>] Option ausschalten (z.B. :nohlsearch)
  \item[hlsearch] Markierung anzeigen
\end{description*}

\subsection*{Register}

\begin{description*}
  \item[\dq ay<Bew>] Fügt <Bew> Text ins Register a
  \item[\dq ap,\dq aP] Fügt Reg. a vor,nach der aktuellen Stelle ein
  \item[\dq\dq, \dq0,\dq<Underscore>] Yank-Register, Delete-Register, <<Black Hole>>
\end{description*}

\subsection*{Spelling}

\begin{description*}
	\item[:set spell] Rechtschreibung einschalten
	\item[:set spelllang=] Sprache einstellen (en\_us, de\_ch)
	\item[z=] Vorschläge anzeigen
	\item[zg] Wort ins Wörterbuch anfügen
	\item[zw] Wort unter dem Curser aus WB löschen
	\item[zug] Rückgängig-Befehl für zg od. zw
\end{description*}

\end{multicols}
\end{document}

